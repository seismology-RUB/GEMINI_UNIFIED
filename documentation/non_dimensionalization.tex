\documentclass[11pt,a4paper]{article}
\usepackage{times}
\usepackage[intlimits]{amsmath}
%
% for reviewers version: remove endfloat
% for original version: keep endfloat
%
\usepackage{graphicx}
\voffset-0.7in
\hoffset-0.4in
\addtolength{\textheight}{5\baselineskip}
\addtolength{\textwidth}{3.0cm}
\addtolength{\oddsidemargin}{-0.5cm}
\renewcommand{\baselinestretch}{1.2}
\pagestyle{myheadings} \markright{}
\begin{document}
\section{Non-dimensionalization of ODEs for a spherically symmetric earth}
%
To compute the seismic displacement in a spherically symmetric Earth, displacement and stress 
are expanded into vector spherical harmonics. Then, the elastodynamic equation of motion reduces 
to a system of ordinary differential equations (SODE) for the expansion coefficients which 
depend on the radial coordinate. The seismic source is represented by the expansion coefficients 
of the force or moment tensor density field. For a point source, we only need to solve a homogeneous
differential equation system. But, the solution must have a jump at the source point:
%
\begin{equation}
\frac{dy}{dr}=A(r)y(r) \quad\mbox{and}\quad y(r_s^+)-y(r_s^-) = s \,,
\end{equation}
%
where $y(r)$ is the displacement-stress vector (DSV), $A(r)$ is the system matrix and $s$ the jump
of the solution at the source radius. For spheroidal motion (P-SV), the DSV has four components 
(if gravity is neglected) $(U,R,V,S)$ where $U$ and $V$ are radial and horizontal displacement, 
respectively, and $R$ and $S$ are normal and tangential stress. A similar two component SODE 
applies for toroidal motion and spheroidal motion in fluid regions. Alternatively, we may write
%
\begin{equation}
\frac{dy}{dr}=A(r)y(r)+s\delta(r-r_s) \,,
\end{equation}
%
where the delta-function has a dimension of inverse length.
%
For non-dimensionalization, we exemplarily consider the spheroidal case in solid regions. 
The matrix $A$ has the form:
%
\begin{equation}
A = \left(\begin{array}{cccc}
-\frac{2F}{rC} & \frac{1}{C} & \ell(\ell+1)\frac{F}{rC} & 0 \\
-\rho\omega^2+\frac{4}{r^2}(A-\frac{F^2}{C}-N) & \frac{2F}{rC}-\frac{2}{r} & -\frac{2\ell(\ell+1)}{r^2}(A-\frac{F^2}{C}-N) & \frac{\ell(\ell+1)}{r} \\
-\frac{1}{r} & 0 & \frac{1}{r} & \frac{1}{L} \\
-\frac{2}{r^2}(A-\frac{F^2}{C}-N) & -\frac{F}{rC} & -\rho\omega^2+\frac{1}{r^2}\ell(\ell+1)(A-\frac{F^2}{C})-\frac{2N}{r^2} & -\frac{3}{r} \\
\end{array}\right) \,.
\end{equation}
%
We introduce a normalization for length, $r_0$, one for density, $\rho_0$, and one for velocity, $v_0$.
Elastic constants are normalized with $\rho_0v_0^2$ and frequency by $v_0/r_0$. The non-dimensional
quantities are marked by a wiggle, e.~g.~$\tilde{F}$. In the same way, we normalize the displacement
stress vector by factors $(1/r_0,1/(\rho_0v_0^2),1/r_0,1/(\rho_0v_0^2))$. We can write this normalization
as a multiplication with a diagonal matrix 
\begin{equation}
D = \mbox{diag}\left(\frac{1}{r_0},\frac{1}{\rho_0v_0^2},\frac{1}{r_0},\frac{1}{\rho_0v_0^2}\right) \,,
\end{equation}
i.~e.~
\begin{equation}
\tilde{y} = Dy \,.
\end{equation}
Expressing the differential equations in terms of the non-dimensional displacement-stress vector, we get:
\begin{equation}
\frac{1}{r_0}\frac{d}{d\tilde{r}}D^{-1}\tilde{y} = AD^{-1}\tilde{y}+D^{-1}\tilde{s}\delta(r-r_s) \,.
\end{equation}
Multiplication from left by $r_0D$ finally leads to
\begin{equation}
\frac{d}{d\tilde{r}}\tilde{y} = r_0DAD^{-1}\tilde{y}+r_0\tilde{s}\delta(r-r_s) \,.
\end{equation}
Since the left hand side is dimensionless, the expression $r_0DAD^{-1}$ must also be dimensionless,
and therefore must be equal to the non-dimensional version of matrix $A$, i.~e.~all dimensioned
elements of $A$ replaced by their non-dimensional equivalents:
\begin{equation}
\tilde{A} = r_0DAD^{-1} \,.
\end{equation}
Explicitly, we find
\begin{displaymath}
\tilde{A} = 
\left(\begin{array}{cccc}
-\frac{2Fr_0}{rC} & \frac{\rho_0v_0^2}{C} & \ell(\ell+1)\frac{Fr_0}{rC} & 0 \\
\frac{r_0^2}{\rho_0v_0^2}\left(-\rho\omega^2+\frac{4}{r^2}(A-\frac{F^2}{C}-N)\right) & \frac{2Fr_0}{rC}-\frac{2r_0}{r} & 
\frac{r_0^2}{\rho_0v_0^2}\left(-\frac{2\ell(\ell+1)}{r^2}(A-\frac{F^2}{C}-N)\right) & \frac{\ell(\ell+1)r_0}{r} \\
-\frac{r_0}{r} & 0 & \frac{r_0}{r} & \frac{\rho_0v_0^2}{L} \\
\frac{r_0^2}{\rho_0v_0^2}\left(-\frac{2}{r^2}(A-\frac{F^2}{C}-N)\right) & -\frac{Fr_0}{rC} & 
\frac{r_0^2}{\rho_0v_0^2}\left(-\rho\omega^2+\frac{\ell(\ell+1)}{r^2}(A-\frac{F^2}{C})-\frac{2N}{r^2}\right) & -\frac{3r_0}{r} \\
\end{array}\right) \,.
\end{displaymath}
Similarly, the expression $r_0\delta(r-r_s)$ is also dimensionless, 
as the delta function has a dimension of inverse length.
Indeed,
\begin{equation}
\int\,\delta(r-r_s)dr = 1 = \int\,\delta(r-r_s)r_0d\tilde{r} \,,
\end{equation}
motivating the definition of the non-dimensional delta-function:
\begin{equation}
\delta(r-r_s)r_0 = \delta(\tilde{r}-\tilde{r}_s) \,.
\end{equation}
Thus, the non-dimensional version of the differential equation system is
\begin{equation}
\frac{d}{d\tilde{r}}\tilde{y} = \tilde{A}\tilde{y}+\tilde{s}\delta(\tilde{r}-\tilde{r}_s) \,.
\end{equation}
%
Note that the non-dimensional angular frequency comes out as 
\begin{displaymath}
\tilde{\omega} = \omega\frac{r_0}{v_0} \,.
\end{displaymath}
The expression $\tilde{s}=Ds$ should be dimensionless, but when looking at the expressions for $s$ in \verb+gemini-displ.tex+,
this seems not to be the case. However, in these expressions, a unit seismic moment or unit seismic force has been omitted.
For a seismic force, the 2nd and 4th components of $s$ are proportional to $F_1/r^2$ where $F_1$ denotes the unit force.
Non-dimensionalizing yields 
\begin{displaymath}
\frac{F_1}{r^2} = \frac{\tilde{F_1}\rho_0v_0^2r_0^2}{\tilde{r}^2r_0^2} = \frac{\tilde{F_1}}{\tilde{r}^2}\rho_0v_0^2 \,.
\end{displaymath}
Since the 2nd and 4th component of $s$ are stress components, they non-dimensionalize properly when the unit force
is taken into account. A similar result is obtained for the moment tensor source terms.
%
\subsection{Non-dimensionalization in GEMINI}
%
GEMINI uses the following implicit non-dimensionalization scheme: $\rho_0 = 1\ \mbox{g}/\mbox{cm}^3$, $v_0 = 1\ \mbox{km}/\mbox{s}$,
and $r_0 = 1\ \mbox{km}$. Thus, $\rho_0v_0^2 = 1\ \mbox{GPa}$. The original units for displacement and stress are hence
km and GPa, respectively. Owing to the implicit non--dimensionalization, it is the non-dimensional force, $\tilde{F}_1$,
which is set to 1. The real force is then $F_1 = \rho_0v_0^2r_0^2 = 10^{15}\ \mbox{N}$. Similarly, the real moment
is $M_1 = \rho_0v_0^2r_0^3 = 10^{18}\ \mbox{Nm}$. The displacement for a single force of $10^{15}\ \mbox{N}$
is hence $\tilde{U}\,\mbox{km}$, or, expressed as displacement per force, $\tilde{U}\cdot 10^{-12}\ \mbox{m}/\mbox{N}$. 
For a moment tensor source, the displacement per moment is $\tilde{U}\cdot 10^{-15}\ \mbox{m}/\mbox{Nm}$.

When GEMINI produces output for computing synthetics or for use with ASKI, it computes values of the non-dimensional
quantities $(\tilde{U},\tilde{R},\tilde{V},\tilde{S})$ and potentially non-dimensional derivatives
$(d\tilde{U}/d\tilde{r},d\tilde{V}/d\tilde{r})$ with meaning as described above for either non-dimensional unit force
or non-dimensional unit moment tensor sources.

To get back to SI units in GEMINI, we should introduce $r_0 = 1\,\mbox{m}$, $\rho_0 = 1\ \mbox{kg}/\mbox{m}^3$, and 
$v_0 = 1\ \mbox{m}/\mbox{s}$. Then, $\rho_0v_0^2 = 1\ \mbox{Pa}$ and the dimensional units for displacement and stress are
m and Pa, respectively. With the non-dimensional force, $\tilde{F}_1 = 1$, the real force is now $F_1 = 1\,\mbox{N}$.
Similarly, for a unit non-dimensional seismic moment, the real moment is $M_1 = 1\,\mbox{Nm}$. Thus, the displacement
for a real force of 1 N or a real moment of 1 Nm is $\tilde{U}\,\mbox{m}$. When using this scheme in a seismological
application, the values of $\tilde{U}$ will become very small. Moreover, a look at the non-dimensional system matrix shows that the
values of the matrix entries will vary by several orders of magnitude potentially corrupting the numerical
accuracy of the integration. For this reason, the non-dimensionalization used in GEMINI seems to make sense and
a rigid adherence to SI units appears to be disadvantageous.

For shallow seismic applications where depths are on the order of meters but velocities still on the order of km/s,
one might use $r_0 = 1\,\mbox{m}$ but still choose $\rho_0 = 1000\,\mbox{kg}/\mbox{m}^3$, and $v_0 = 1000\,\mbox{m}/\mbox{s}$.
In that case, values of non-dimensional frequency $\tilde{\omega}$ no longer represent the unit Hz.
Instead, for a real angular frequency of 1 kHz, we obtain now $\tilde{\omega} = 1$. Non-dimensional frequency hence
represents the unit kHz! The real force decreases to $F_1 = \rho_0v_0^2r_0^2 = 10^9\,\mbox{N}$ and displacement per force
is $\tilde{U}\cdot 10^{-9}\,\mbox{m}/\mbox{N}$.

In general, displacement per force in unit m/N can be calculated according to the formula 
\begin{displaymath}
\mbox{displacement per force} = \tilde{U}\frac{r_0}{\rho_0v_0^2r_0^2} = \tilde{U}\frac{1}{\rho_0v_0^2r_0} \,,
\end{displaymath}
where $\rho_0$, $v_0$, $r_0$ must be given in SI units. Displacement per moment in unit m/(Nm) follows the formula
\begin{displaymath}
\mbox{displacement per moment} = \tilde{U}\frac{r_0}{\rho_0v_0^2r_0^3} = \tilde{U}\frac{1}{\rho_0v_0^2r_0^2} \,,
\end{displaymath}
%
\subsection{Unit jump Green functions}
%
The definition of unit jump Green functions only makes sense if all quantities are non-dimensional. We set the non-dimensional
jump vector to a unit vector pointing in the $l$-direction, $\tilde{s} = e_l$, and obtain the $l$-th unit jump Green functions
as a solution of
\begin{equation}
\frac{d}{d\tilde{r}}G_l(\tilde{r},\tilde{r_s}) = \tilde{A}G_l(\tilde{r},\tilde{r_s})+e_l\delta(\tilde{r}-\tilde{r}_s) \,.
\end{equation}
Hence, all the derivations given in \verb+rezifrechet.tex+ apply to the non-dimensional unit jump Green function.
%
In particular, we have the reciprocity relation:
\begin{equation}
\tilde{r}_s^2 G_k^i(r_s,r_e) T_{il} = \tilde{r}_e^2 G_l^i(r_e,r_s) T_{ik} \,.
\end{equation}
Moreover, the expression for the Frechet derivative of the stress-displacement vector is:
\begin{equation}
\delta \tilde{y}(\tilde{r}_e) = -\sum_d \tilde{h}_d G_k(\tilde{r}_e,\tilde{r}_d)
\left(\left[\tilde{A}(\tilde{r}_d)\right]^+_- \tilde{y}(\tilde{r}_d)\right)_k 
+\int G_k(\tilde{r}_e,\tilde{r})(\delta \tilde{A}(\tilde{r}) \tilde{y}(\tilde{r}))_k \,d\tilde{r} \,.
\end{equation}
Identifying the DSV-vector with the unit jump Green function for a source at $r_s$, we obtain,
\begin{equation}
\delta G_l(\tilde{r}_e,\tilde{r}_s) = -\sum_d \tilde{h}_d G_k(\tilde{r}_e,\tilde{r}_d)
\left(\left[\tilde{A}(\tilde{r}_d)\right]^+_- G_l(\tilde{r}_d,\tilde{r}_s)\right)_k 
+\int G_k(\tilde{r}_e,\tilde{r})(\delta \tilde{A}(\tilde{r}) G_l(\tilde{r},\tilde{r}_s))_k \,d\tilde{r} \,.
\end{equation}
Expressed in component notation, we obtain:
\begin{equation}
\delta G^i_l(\tilde{r}_e,\tilde{r}_s) = -\sum_d \tilde{h}_d G^i_k(\tilde{r}_e,\tilde{r}_d)
\left(\left[\tilde{A}_{kj}(\tilde{r}_d)\right]^+_- G^j_l(\tilde{r}_d,\tilde{r}_s)\right)_k 
+\int G^i_k(\tilde{r}_e,\tilde{r})(\delta \tilde{A}_{kj}(\tilde{r}) G^j_l(\tilde{r},\tilde{r}_s))_k \,d\tilde{r} \,.
\end{equation}
\end{document}
